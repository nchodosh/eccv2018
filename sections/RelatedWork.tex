\section{Related Work}
\label{sec:related-work}

Since our proposed method is a fusion of deep learning and compressed sensing, we will review in this section previous work that has used either technique for depth estimation.

\subsection{Compressed Sensing}
\label{sec:compressed-sensing}

% Hawe et al - Single layer wavelet dictionaries for dense disparty maps
% L.-K.Liu et al - More complicated hand crafted dictionaries
% Ma \& Karaman - Use indoor planar assumptions for recovery

Compressed sensing is a technique in signal processing for recovering signals from a small set of measurements. Naturally it has been applied to depth completion in previous work, but has been limited to single-level hand-crafted dictionaries. The earliest is Hawe \etal, who show that disparity maps can be represented sparsely using the wavelet basis\cite{hawe2011dense}. L.-K. Liu \etal built on that by combining wavelets with contourlets and investigated the effect of different sampling patterns\cite{liu2015depth}. Both methods were out performed by Ma \& Karaman who exploit the simple structures of man-made indoor scenes to achieve full depth reconstruction\cite{ma2016sparse}. In contrast to all of these works, our approach learns multi-level convolutional dictionaries from a large dataset of incomplete ground truth depth maps.

\subsection{Deep Learning}
\label{sec:depth-upsampling}
% Eigen et al - first to use deep learning for rgb to depth 
% F. Liu et al - Fuse deep network with CRF
% Lania \etal - Introduced up projection blocks for rgb to depth
% Ma \& Karaman - Very deep network for depth prediction
% SparseCNN - Similar to ours in that it explicitly handles sparsity

Depth estimation using deep learning has largely been restricted to single-shot, RGB to depth prediction. This line of inquiry started with Eigen \etal who showed that a deep network could reasonably estimate depth using only an RGB image \cite{eigen2014depth}. Many variants of this method have since been explored\cite{kuznietsov2017semi,godard2016unsupervised,liu2015deep}. Lania \etal introduced up-projection blocks which allowed for very deep networks and several other works have proposed variants of their architecture ~\cite{laina2016deeper}. The most relevant of these variants is the Sparse-to-Dense network of Ma \& Karaman, which also examines depth completion from LiDAR points\cite{sparsetodense}. Urhig \etal introduced the KITTI depth completion dataset, and showed that CNNs which explicitly encode the sparsity of the input achieve much better performance\cite{uhrig}. Riegler \etal designed ATGV-Net, a deep network for depth map super resolution, but they assume a rectangular grid of inputs so it is not applicable to LiDAR completion\cite{riegler}. We will use the methods of Ma \& Karaman and Urhig \etal as our baseline comparisons since they represent the state-of-the-art in LiDAR depth completion.